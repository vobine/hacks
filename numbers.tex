\documentclass[twocolumn]{article}

\usepackage{amsmath}

\usepackage{amsthm}
\newtheorem{lemma}{Lemma}
\newtheorem{theorem}{Theorem}

\newcommand{\PrimeSet}{\mathbf{P}}

\title{Number Theory}
\author{Haldane Peterson}

\begin{document}
\maketitle

\section{Why are we here?}

For Dad!
And my own amusement.

\section{Unbounded Primes}

There is no bound on the number of primes.
To prove it, we start with a property of factoring that will be
useful:

\begin{lemma}
\label{lemma:next-composite}
If a composite number $c$ has a prime $p$ as a factor, i.e.,
\begin{equation}
c = a p
\label{eqn:cap}
\end{equation}
for some integer~$a$, then $c+1$ \emph{cannot} be a multiple of~$p$.
\end{lemma}

To prove Lemma~\ref{lemma:next-composite}, assume the opposite:  that
\begin{equation}
c + 1 = b p
\label{eqn:c1bp}
\end{equation}
for some integer~$b$.
We can subtract equation~\ref{eqn:cap} from
equation~\ref{eqn:c1bp} to get:
\begin{eqnarray*}
(c + 1) - c & = & b p - a p \\
1 & = & p (b - a) \\
\end{eqnarray*}
So:
\begin{equation}
\label{eqn:b-a}
1 / p = b - a.
\end{equation}

By definition, $p > 1$, so:
\begin{equation*}
0 < 1 / p < 1
\end{equation*}
However, the difference $b-a$ is an integer, because~$b$ and~$a$ are
both integers.
The upshot is that
\begin{equation}
\label{eqn:b-a-neq}
1 / p \neq b - a.
\end{equation}
Equation~\ref{eqn:b-a-neq} directly contradicts
Equation~\ref{eqn:b-a},
therefore Equation~\ref{eqn:cap} and Equation~\ref{eqn:c1bp} cannot
both be true.
So, by contradiction, Lemma~\ref{lemma:next-composite} must be true.
\qed

Now for the main event:
The number of primes is infinite, or, equivalently:
\begin{theorem}
\label{theorem:unbounded-primes}
For any prime~$p$ there exists a larger prime~$p'$.
\end{theorem}

We prove this by contradiction.
Suppose, \textit{contra} Theorem~\ref{theorem:unbounded-primes}, that
there is a finite number of primes,~$n$.
Then the largest prime is $p_n$ and the set of all primes is:
\begin{equation*}
\PrimeSet = \{ p_1, p_2, \ldots, p_n \},
\end{equation*}

We can multiply all of the primes to construct a (very
large) composite number~$c$:
\begin{equation}
c = p_1 \cdot p_2 \cdot p_3 \cdots p_{n-1} \cdot p_n.
\end{equation}

Now consider $c+1$.
By Lemma~\ref{lemma:next-composite}, none of the prime factors of~$c$
can be a factor of~$c+1$.
So one of two cases must hold:
\begin{itemize}
\item $c+1$ is itself a prime.
Since $c+1$ is greater than any $p_i \in \PrimeSet$,
then
\begin{equation}
c+1 \notin \PrimeSet.
\end{equation}
That contradicts the assumption that~$\PrimeSet$ is exhaustive.
\item $c+1$ is composite.
But Lemma~\ref{lemma:next-composite} established that none of
the~$p_i$ are factors of $c+1$.
Therefore the prime factors of $c+1$, whatever they are, must not be
in~$\PrimeSet$.
Again, $\PrimeSet$~is not exhaustive, contradicting the assumption.
\end{itemize}

Therefore no finite~$\PrimeSet$ \emph{can} be exhaustive. The primes are
infinite. \qed

\subsection{History and Humor}

I adapted this section from memory and from a description on Proof
Wiki\footnote{proofwiki.org.} of Euclid's Theorem.
The theorem dates back to at least 300~BCE, as Proposition~20 in
\textit{Elements}, Book~IX.
I haven't yet tracked down an attribution for the proof.  Is it from
Euclid or a later author?  Time to find a copy of \textit{Elements}.

A student at a conference asks: ``Are all odd numbers prime?''
A mathematician answers, ``3, 5, and~7 are prime, but 9~is
not.  So no.''
A physicist answers, ``Yes!  3~is prime, 5~is prime, 7~is prime, 9~is
experimental error, 11~is prime, etc.''
An engineer answers, ``Yes!  3~is prime, 5~is prime, 7~is prime, 9~is
prime, \ldots.''

%% \section{How Many Rationals?}

%% \section{Are All Real Numbers Rational?}

%% \section{How Many Real Numbers?}

\appendix

\section{Notations and Proofs}

\end{document}
